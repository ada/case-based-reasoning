\section{Summary and conclusions}
The developed classifier met all the requirements which were validated
by the various tests performed. Audio files of eight different birds has been
acquired while the birds are singing. A moving window
with a predefined boundaries scans the signal in the time domain and a tranformation
from the time-domain to frequency domain has been conducted in order to extract features.
The two frequencies with the highest amplutide have been extracted for each frame.
These two features together with the class constitute the database for CBR.

The CBR written in the Ada programming language uses the K nearest approach to
find the most similar cases and the user can specify the similarity formula with ease.
The leave-one-out cross-validation method has been used for the validation of the CBR system.

A natural evolution of this classifier would be to use more appropriate features
and adjusting the feature weights accordingly. Also remvoing duplicate data from
the the database would increase the precision of the system and when that happens
the user can test for classification of other animals such as whales or dogs.

With higher reliability, this system could be used for people with hearing impairment
in order to hear the sound of different subjects by using vibration motors. 
