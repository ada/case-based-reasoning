\section{Introduction}
Bird song classification is not only useful for enthusiastic bird watchers.
It contributes to research fields such as ornithology and ecology \cite{7418412},
for example conservatory planning  \cite{5360230} and the understanding of
how climate change affects bird populations \cite{7418412}.
It is also useful for preventing plane crashes caused by collisions with birds  \cite{5360230}.

Singing serves different purposes for birds. For example identify
individuals and kinship, threaten rivals and choose partner  \cite{7073944}.
Just like human speech, bird song have a grammar and are built on syllables \cite{5360230}.
Some birds hide and therefore sound is the only way for humans to know their presence.
Study of birds by audio recording is less invasive than humans present in the nature \cite{6123334}.

Since automated bird song classification can be more efficient and accurate than manual,
research has been done. One of the challenge is that the recordings are naturally
noisy and can contain sound from many individuals at the same time  \cite{7073944}.
Extraction of useful features is crucial \cite{5360230}, as well as the selection
of classification algorithm.

This paper presents the development of audio-based bird specie classification
using Case-based reasoning (CBR) \textit{K} Nearest Neighbor (kNN) as well as
extracting features from audio recordings of several birds including: Bluethroat,
Bluethroat2, Chaffinch, Wood Warbler, Cuckoo, Blackbird, Nightingale, Redbreast and Great Tit.

After creating the database and the classifier, an additional audio recording
of a bluethroat bird (called bluethroat2) has been added and validated against the database.

Section two describes the background and related works on CBR, feature extraction and audio fingerprinting.
Section three contains the detailed implementation of feature extraction and kNN. Experimental
result on different feature sets are listed in section IV.
