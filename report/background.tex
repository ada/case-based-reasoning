\section{Background and related work}
Robert Gubkar and Michal Kuba\cite{6566278}  from the university of Zilinia in Slovakia
conducted a study about comparison of audio features for elementary sound
based audio classification. They compared two sets of audio features for classification
of various sounds such as applause, crying, laugh, music, noise and speech.
The first set of features consist of mel-frequency ceptral coefficients together with
their first and second order time derivatives. The seconds set consists of line
spectral frequencies, spectral flux and zero crossing rate. Their test shows that
the second set of features were superior to classical MFCC coefficients for general
sound pattern recognition and comparable for speech recognition.\\

Jijung Deng et. al.\cite{6138136} created a fingerprinting system based on spectral energy
structure of the audio signal. Audi fingerprint is a compact unique content-based
digital signature which is constructed based on features from the audio.
The used feature extraction method consists of filtering out the audio signal
to eliminate high-frequency noise and fourier-transform is applied to calculate
the energy of the subbands. Each subband energy is represented by 2 bits.
At last a sub-fingerprint of 2*(number of subbands-1) bits in binary form will
be produced from each frame of audio signal. The usage of inverted file
indexing technique combined with hash tables increased the speed of retrieval
process for the matching algorithm. Their preliminary experimental results
suggest that this method can work well in the application of broadcast
monitoring.\\


Identifying sounds using computers is not something new, the idea has been
around for a long time. So several different methods for this has been
developed, in this project Periodogram was used for feature extraction and
Case Based Reasoning was used for the decision making part.\\

This paper from 2014\cite{danne1} instead decided to use Spectrogram combined with
Angular and Radial Transform (ART) for their feature extraction part. A
Spectrogram is a 3D representation of a sound file. This gives a lot of
potential features to extract but it is a lot harder to find and extract
these features than from a 2D graph. They solve this by using ART which is a
moment based image description method. This is then adopted in MPEG-7 as a
region-based shape descriptor, this gives an efficient way of expressing pixel
distribution in a 3D region. Essentially the 3D graph will instead be thought
of as a 3D image and this will allow the ART to analyze the shape and pixels,
this can be used to extract the features from the spectrogram. This is then
applied much in the same way we did with the sound file being analyzed with a
texture window creating overlapping frames where the features are extracted in
every frame. For their decision making they instead use Gaussian Mixture Models
(GMM) which is parametric probability density function. This method differs
from CBR in that you don't build a library of different cases and then compare
the current case to your library. Instead a GMM needs to be trained beforehand
and then uses weights to help in the decision making. In this case they used
Expectation-Maximization algorithms, each bird species will be represented as
a combination of Gaussian Component probability density functions.\\

Another paper that is not directly related to the work done in this project,
but is very important to the area of signal identification as a whole is\cite{1566472}.
Reconstruction of incomplete spectrograms for robust speech recognition, this
paper details a method for gaining a more robust recognition of speech when the
sound is corrupted by noise. When a signal is corrupted by noise some features
may be lost, normally a program would be trained with cases that have a similar
noise level. This methods ignore the lost features and use the remaining to
find the most likely match, this is by no means a foolproof method and this
only works with stationary noise. Non-stationary noise is even more error prone.
 This paper instead talks about the reconstruction of the corrupted data by
 estimating the missing components and after the data has been estimated the
 signal can be reconstructed and then classified with the speech recognition
 program. The paper details several different methods for reconstructing
 incomplete or corrupted spectrograms, either by treating the missing regions
 as completely unknown or treating them as unknown but bounded. Several
 different methods are introduced but the two most effective ones are cluster
 marginal reconstruction and covariance joint reconstruction.\\



Paper\cite{6497946} present a method to reduce the number of fingerprints or cases by increasing the fingerprint size which will improve memory consumption and search time.
"The proposed method uses the fingerprint extraction algorithm in time and frequency domains using wavelet transform. The audio signal was decomposed into 5-layer wavelet, and then calculated the energy distribution center, the energy of sub-band in wavelet domain, and the variance of wavelet coefficient. Finally, by using the results Calculated as the parameters of the audio fingerprints, the 8-bit fingerprint block per frame was generated"\cite{6497946}.\\


Chieh-Yuan Tsai Et. al. describe Case-based reasoning as a problem-solving technique
commonly seen in artificial intelligence\cite{5169353}. "The K-nearest neighbor (KNN) search
method which selects the K most similar prior cases for a new
case has been extensively used in the case retrieval phase of
CBR. Although KNN can be simply implemented, the choice of
the K value is quite subjective and will influence the
performance of a CBR system."
