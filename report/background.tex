\section{Background and related work}
Robert Gubkar and Michal Kuba \cite{6566278}  from the university of Zilinia in Slovakia
conducted a study about comparision of audio features for elementary sound
based audio classification. They compared two sets of audio features for classification
of various sonds such as applause, crying, laugh, music, noise and speech.
The first set of features consist of mel-frequency ceptral coefficients together with
their first and seconds order time derivaties. The seconds set consists of line
spectral frequencies, spectral flux and zero crossing rate. Their test shows that
the second set of features were superior to classical MFCC coefficients for general
sound patter recognition and comparable for speech recognition.

Jijung Deng et. al.\cite{6138136} created a fingerprinting system based on spectral energy
structure of the audio signal. Audi fingerprint is a compact unique content-based
digital signature which is constructed based on features from the audio.
The used feature extraction method consists of filtering out the audio signal
to eliminate high-frequency noise and fourier-transform is applied to calculate
the energy of the subbands. Each subband energy is represented by 2 bits.
At last a sub-fingerprint of 2*(number of subbands-1) bits in binary form will
be produced from each frame of audio signal. The usage of inverted file
indexing technique combined with hash tables increased the speed of retrieval
process for the matching algorithm. Their preliminary experimental results
suggest that this method can work well
in the application of broadcast monitoring.
