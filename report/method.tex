\section{Method}
To make our bird recognition program we would essentially need two different parts. First a feature extraction method that can analyze and extract the desired features from a sound file of a bird singing. Secondly a decision making algorithm that can take the extracted features and classify which bird that sings in the sound file. The methods we choose for feature extraction was Periodogram (spectral density estimation) and CBR (Case Based Reasoning) was used for the decision making, we choose CBR specifically because it’s a tried and true decision making method. 

Periodogram
So for our project we used Periodogram to step through the sound file one frame at a time and then get the frequencies and corresponding amplitude in each frame. We did this using matlab. The sound file is first loaded into the program and the frame size is set depending on the length of the sound file, the step size is preset. After that the program basically uses the “Periodogram” function already present in matlab to create a periodogram of the frame illustrated in Fig.1. After that the “Findpeak” function is used to find the two highest peaks in the frame see Fig.2, this corresponds to the two frequencies that have the highest amplitude. One of the biggest reasons we choose Periodogram for our feature extraction was because of the ease of which you can find peaks in a two dimensional graph. This information along with the mean value and standard deviation is stored in an array. This is then done for the next frame and so on until the entire sound file has been analyzed as can be seen in Fig.3. This array is then exported as a csv file that will be used in the CBR program. To create the case library for the CBR several sound files of different birds are analyzed and exported into csv files.

CBR
The files created with the Periodogram are then put where the CBR can find them. So when the CBR runs you have to tell it which of the frames you want in to test, the program will then compare the features in this frame to the cases that are already in the library. The features which are relevant for our program are the two frequencies with the highest amplitude, in this case it the decibel value is the amplitude. The program will use the KNN (K-nearest neighbour) algorithm to determine which bird made the sound. We have tried several different similarity functions but the one we decided to use in our final program was Euclidian distance simply because it gave us the best performance. Only two features is usually not enough to get a robust algorithm, so the program will need another frame from the same source and will also run this through the KNN algorithm. The results from the two frames are then compared and the the class with the most neighbours from both of the frames is chosen see Fig.4.
