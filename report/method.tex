\section{Method}
To make a bird recognition program two main parts are needed. First a feature
extraction method that can analyze and extract the desired features from a
sound file of a bird singing. Secondly a decision making algorithm that can
take the extracted features and classify which bird that sings in the sound
file. The method choosen feature extraction was Periodogram (spectral density
estimation) and CBR (Case Based Reasoning) was used for the decision making,
CBR was chossen specifically because it’s a tried and true decision making method.

\section{Feature extraction}
Periodogram was used to step through the sound file one frame at a time and
then get the frequencies and corresponding amplitude in each frame. This was
done using matlab, the sound file is first loaded into the program and the
frame size is set depending on the length of the sound file, the step size
is preset. After that the program basically uses the “Periodogram” function
already present in matlab to create a periodogram of the frame illustrated in
Fig.1. After that the “Findpeak” function is used to find the two highest
peaks in the frame see Fig.2, this corresponds to the two frequencies that have
the highest amplitude. The main reason for the use of periodogram for the
feature extraction was the ease that peaks can be identified in a two
dimensional graph. This information along with the mean value and standard
deviation is stored in an array. This is then done for the next frame and
so on until the entire sound file has been analyzed as can be seen in Fig.3.
This array is then exported as a csv file that will be used in the CBR program.
To create the case library for the CBR several sound files of different birds
are analyzed and exported into csv files.

\section{Case matching algorithm}
The files created with the Periodogram are then put where the CBR can find them.
Before the CBR can be run, the frames that are to be tested have to be
speciefied, the program will then compare the features in this frame to the
cases that are already in the library. The features which are relevant for
the program are the two frequencies with the highest amplitude, in this case
it the decibel value is the amplitude. The program will use the KNN
(K-nearest neighbour) algorithm to determine which bird made the sound.
Several different similarity functions have been tried and for this program
Euclidian distance gave the best performance. Only two features is usually
not enough to get a robust algorithm, so the program will need another frame
from the same source and will also run this through the KNN algorithm. The
results from the two frames are then compared and the the class with the most
neighbours from both of the frames is chosen see Fig.4.
